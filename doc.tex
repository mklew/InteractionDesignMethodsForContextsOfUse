% Created 2013-11-12 Tue 20:00
\documentclass[12pt, a4paper]{article}
\usepackage[utf8x]{inputenc}
\usepackage{geometry}
\usepackage{hyperref}
\usepackage{amsmath}
\usepackage[numbers]{natbib}
\usepackage{algorithm}
\usepackage{algpseudocode}
\usepackage{float}
\usepackage{graphicx}
\usepackage{listings}

\author{Marek Lewandowski \\ Jérémy Bossut}
\date{\today}
\title{Interaction Design - Methods for the specification of the context of use}
\begin{document}

\maketitle
\newpage

\section{Descriptions of methods}
% –Description (with references). Students should search for additional sources, from best to less interesting: books, research articles, and web pages.

\subsection{Description of cultural probe/diary study}


Note: There are minor differences between a cultural probe and a diary study as a cultural probe is supposed to encourage imaginative reflection while diary study is only about gathering descriptive facts. Here, we consider both terms as synonyms.
 
As Suzanne Ginsburg states, “diary studies shift the burden of data collection onto the participant” \cite{ginsburg_designing_2010}. With this requirement elicitation technique, researchers can get information about people's activity and experience on a particular subject over time without leaving their workplace. This kind of study consists in giving to observers a kit of some materials including a diary and letting them report their actions.

This method is particularly convenient when you need to collect data over a long period of time or for situations when direct observation would be impractical or disturbing. For example, if you want to know more about how sailors manage their time on a boat, a diary study is a good choice. It’s also very useful for studying the use of a mobile application as they are mainly used intermittently.

Diary studies offer many ways of gathering information \cite{webcredible}. In most cases, the diary is a paper booklet with entry forms but it can also be done by voice messaging or online applications such as emails, a private Twitter account or a custom website or mobile application. Each platform has its own advantages and drawbacks and has to be carefully thought according to the subject of the study. A cultural probe kit also includes very frequently a disposable camera to document reports with pictures. This kit needs to look attractive and professional to encourage participants to do a good job.

Since diary studies cannot be monitored closely during their execution, it is important to spend time on preparation and recruitment process. According to authors of Observing the User Experience \cite{goodman2012observing}, a typical diary study has about 10 participants and a careful researcher should recruit participants with a focus on reliability over availability. Besides, recruiting extra people is advised as it is likely to observe some dropouts. A lot of effort is required from participants and that’s why they need to be paid.

To ensure a good quality of gathered data, participants have to be properly briefed about how to report and what information do researchers need. Once it has begun, it is essential to keep in touch with participants throughout the study \cite{eriontheinterweb} and follow-up interviews are often conducted to help clarify user behavior. Ending the study with an interview is also a good way to validate information gathered. 


\subsection{Description of photo study}
Photo study is another method of requirements elicitation. This method can be applied in short-term study so that results can be obtained in for example two weeks which is much less compared to cultural probe which might take months.

Basic idea is as follows. Participants receive camera and some mission to complete. They complete the mission by taking pictures relevant for the mission. After scheduled period, participants return cameras with photos for the researchers who conduct photo analysis. Analysis might reveal custmer requirements. For example participants in Edmonton Public Library's study\cite{photostudyExample} had to capture their perception of library space. In the result pictures taken by participants gave insight about how customers use those spaces. Author of the study said that ,,findings revealed the comprehensive views participants’ hold about the library; the 
library’s spaces are not distinct from the collections and services offered within them.'' 

Photo study is conducted by unsupervised participants. That means that participants have to be recruited in the first place. Finding right candidates might be a tough step in the study. Study is unsupervised which means that there is no way to oversee participants during the study. It can also be advantage over other methods because photos might be taken in inaccessible and private places capturing sensitive information and participants lifestyle. On the other hand there is risk that participants return with photos which do not convey much of the useful information. However experience shows that this method is cost effective and leads to discovery of requirements and previously unknown customer needs.

After participants return with the photos another phase of the study begins. It is analysis. During the analysis of the photos researchers try to find common themes, problems, needs, requirements and design ideas. Depending on the work of the participants there might be a lot of data to process because taking digital pictures is very cheap.

Suzanne Robertson and James Robertson say that ,,Photos are an effective way of giving people a detailed review without having to write a huge report''. \cite{robertson2006mastering} This is another approach to the photo study where photos are taken during every interview, meeting, workshop. Basically everything that could be gone after the meeting is being captured by camera so it is possible to go back to that meeting or workshop or office and once again take a look at what happened. That is useful because knowledge obtained from the meeting is gone pretty soon after the meeting unless it is somehow preserved. Creating reports might be time consuming process and taking photos is very easy and fulfils the same purpose.




\section{Case Study - Gardener Assistant - mobile application}
\section{The problem}
Gardener has to take notes about sick plants and other issues in the natural park. Creating these notes is time consuming process. Written notes are also not easily manageable. There are also other problems with taking paper notes in external environment for example when it rains.

\subsection{Cultural probe plan}
Conducting cultural probe requires to prepare kit and plan for the study.

Contents of the cultural probe kit:
\begin{itemize}
\item instant camera
\item diary
\item pens
\item contact information for asking questions during the study
\item glue
\end{itemize}

Detailed plan of the study:
\begin{enumerate}
\item Find appropriate gardeners for the study. The right gardener should understand the purpose of the study and be willing to spend his time during 1 month study;
\item Brief gardener about the purpose and expectations of the study. Gardener should document his feelings, interactions and his work as well as the others by taking notes and pictures;
\item Provide gardener with cultural probe kit;
\item Conduct follow-up interview after 1 week to ensure gathered data is relevant;
\item Collect materials after 1 month from the begining of the study;
\item Analyse materials;
\item Conduct de-briefing session to validate, explore information gathered by gardener;
\item Analyse all information gathered;
\item Prepare document with results of analysis;
\item Use documents to elicit requirements about gardener assistant application.
\end{enumerate}

\subsection{Photo study plan}
Mission of the photo study is that gardener should take pictures of his own work as well as work of the others. Pictures should reflect on issues that gardener stumbles upon. TODO
\begin{enumerate}
\item Give camera to gardener;\footnote{or more given larger budget}
\item Receive photos from gardener;
\item Analyse photos:
  \begin{itemize}
    \item look for common themes,
    \item look for opportunities for new technologies,
    \item look for technology barriers.
  \end{itemize} 
 \item Prepare document with results of analysis;
 \item Use report to elicit requirements.
\end{enumerate}

\section{Critical analysis of the methods}
% Critical analysis of the degree of difficulty for its application: Analysis on the difficulty of application of the method (resources needed and knowledge required), from a software engineer point of ovew.

In both cases the most difficult part is finding the right person for the study. In both methods client is responsible for the success of the study. Software engineers do not have close control over participants. The only form of control is the follow-up interview during the cultural probe study. 

Difficulty varies. In case of cultural probe preparation of the cultural probe kit is a more difficult task than just giving someone camera. Contents of the kit are important because they directly affect the results of the whole study. 

Cultural probe is kind of an extension of the photo study.

\section{Results}
Application helps gardeners to do reports about issues in the natural park for example sick plants, not working traps, damages of different kind. Each report requires note and location coordinates. Application can ease the process of creating the report by automatically gathering some information like coordinates using geo-location technique, time of report. Gardener can attach photos taken from phone's camera at the spot. Application allows to take audio notes as well as usual text notes.

\nocite{*}
\bibliographystyle{plainnat}
\bibliography{bibliography}
\end{document}