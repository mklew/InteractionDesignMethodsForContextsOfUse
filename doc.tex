% Created 2013-11-12 Tue 20:00
\documentclass[12pt, a4paper]{article}
\usepackage[utf8x]{inputenc}
\usepackage{geometry}
\usepackage{hyperref}
\usepackage{amsmath}
\usepackage[numbers]{natbib}
\usepackage{algorithm}
\usepackage{algpseudocode}
\usepackage{float}
\usepackage{graphicx}
\usepackage{listings}

\author{Marek Lewandowski \\ Jérémy Bossut}
\date{\today}
\title{Interaction Design - Methods for the specification of the context of use}
\begin{document}

\maketitle
\newpage

\section{Descriptions of methods}
% –Description (with references). Students should search for additional sources, from best to less interesting: books, research articles, and web pages.

TODO

\subsection{Description of cultural probe}

\subsection{Description of photo study}


\section{Case Study - Gardener Assistant - mobile application}
\section{The problem}
Gardener has to take notes about sick plants and other issues in the natural park. Creating these notes is time consuming process. Written notes are also not easily manageable. There are also other problems with taking paper notes in external environment for example when it rains.

\section{Results}
Application helps gardeners to do reports about issues in the natural park for example sick plants, not working traps, damages of different kind. Each report requires note, photo and location coordinates. Application can ease the process of creating the report by automatically gathering some information like geo-location, time. Gardener can attach photos taken from phone's camera at the spot. Application allows to take audio notes as well as usual text notes.


\nocite{*}
\bibliographystyle{plainnat}
\bibliography{bibliography}
\end{document}
