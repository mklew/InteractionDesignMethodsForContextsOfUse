% Created 2013-11-12 Tue 20:00
\documentclass[12pt, a4paper]{article}
\usepackage[utf8x]{inputenc}
\usepackage{geometry}
\usepackage{hyperref}
\usepackage{amsmath}
\usepackage[numbers]{natbib}
\usepackage{algorithm}
\usepackage{algpseudocode}
\usepackage{float}
\usepackage{graphicx}
\usepackage{listings}

\author{Marek Lewandowski \\ Jérémy Bossut}
\date{\today}
\title{Interaction Design - Methods for the specification of the context of use}
\begin{document}

\maketitle
\newpage

\section{Descriptions of methods}
% –Description (with references). Students should search for additional sources, from best to less interesting: books, research articles, and web pages.

\subsection{Description of cultural probe}

Prepare appropriate cultural probe kit.

\begin{enumerate}
\item Find appropriate candidates for the study
\item Brief candidates about the purpose and expectations of the study
\item Provide candidate with cultural probe kit
\item Conduct follow-up interview to ensure everything is done correctly
\item Collect materials
\item Analyse materials
\item Conduct de-briefing session to validate, explore information gathered by candidates
\item Analyse all information gathered
\item Prepare document with results of analysis
\item Use documents to elicit requirements
\end{enumerate}

\subsection{Description of photo study}


\section{Case Study - Gardener Assistant - mobile application}
\section{The problem}
Gardener has to take notes about sick plants and other issues in the natural park. Creating these notes is time consuming process. Written notes are also not easily manageable. There are also other problems with taking paper notes in external environment for example when it rains.

\subsection{Cultural probe plan}
Conducting cultural probe requires to prepare kit and plan for the study.

Contents of the cultural probe kit:
\begin{itemize}
\item instant camera
\item diary
\item pens
\item contact information for asking questions during the study
\item glue
\end{itemize}

Detailed plan of the study:
\begin{enumerate}
\item Find appropriate gardeners for the study. The right gardener should understand the purpose of the study and be willing to spend his time during 1 month study;
\item Brief gardener about the purpose and expectations of the study. Gardener should document his feelings, interactions and his work as well as the others by taking notes and pictures;
\item Provide gardener with cultural probe kit;
\item Conduct follow-up interview after 1 week to ensure gathered data is relevant;
\item Collect materials after 1 month from the begining of the study;
\item Analyse materials;
\item Conduct de-briefing session to validate, explore information gathered by gardener;
\item Analyse all information gathered;
\item Prepare document with results of analysis;
\item Use documents to elicit requirements about gardener assistant application.
\end{enumerate}

\subsection{Photo study plan}
Mission of the photo study is that gardener should take pictures of his own work as well as work of the others. Pictures should reflect on issues that gardener stumbles upon. TODO
\begin{enumerate}
\item Give camera to gardener;\footnote{or more given larger budget}
\item Receive photos from gardener;
\item Analyse photos:
  \begin{itemize}
    \item look for common themes,
    \item look for opportunities for new technologies,
    \item look for technology barriers.
  \end{itemize} 
 \item Prepare document with results of analysis;
 \item Use report to elicit requirements.
\end{enumerate}

\section{Critical analysis of the methods}
% Critical analysis of the degree of difficulty for its application: Analysis on the difficulty of application of the method (resources needed and knowledge required), from a software engineer point of ovew.

In both cases the most difficult part is finding the right person for the study. In both methods client is responsible for the success of the study. Software engineers do not have close control over participants. The only form of control is the follow-up interview during the cultural probe study. 

Difficulty varies. In case of cultural probe preparation of the cultural probe kit is a more difficult task than just giving someone camera. Contents of the kit are important because they directly affect the results of the whole study. 

Cultural probe is kind of an extension of the photo study.

\section{Results}
Application helps gardeners to do reports about issues in the natural park for example sick plants, not working traps, damages of different kind. Each report requires note and location coordinates. Application can ease the process of creating the report by automatically gathering some information like coordinates using geo-location technique, time of report. Gardener can attach photos taken from phone's camera at the spot. Application allows to take audio notes as well as usual text notes.

\nocite{*}
\bibliographystyle{plainnat}
\bibliography{bibliography}
\end{document}
